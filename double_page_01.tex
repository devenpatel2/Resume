% LaTeX file for resume 
% This file uses the resume document class (res.cls)

\documentclass{res} 
\usepackage[top=10mm, left=15mm]{geometry}

%\usepackage{helvetica} % uses helvetica postscript font (download helvetica.sty)
%\usepackage{newcent}   % uses new century schoolbook postscript font 
\setlength{\textheight}{10.5in} % increase text height to fit on 1-page 

\begin{document} 

\name{Deven Patel\\[12pt]}     % the \\[12pt] adds a blank
				        % line after name      
\address{deven.patel2@gmail.com\\\hspace{6mm}+919819482499}         
\begin{resume}
\begin{center}
\line(1,0){400}
\end{center}
 
\section{PROFILE}
 A Computer Vision post graduate with  experience as a developer as well as a
 researcher in the field of Image Processing and Computer Vision.  Proficient
 with OpenCV  on Linux as well as Android platforms and have extensively worked on
 relevant projects. 
 \section{EXPERIENCE}
 
   \vspace{-0.05in}	
   \begin{tabbing}
   \hspace{2.3in}\= \hspace{2.6in}\= \kill % set up two tab positions
   
    {\bf Image Processing Consultant}     \> \> Sep 2014 - Jan 2016\\
        Binaryveda Software Ltd. \>  \>CBD-Belapur, Navi Mumbai. 
   \end{tabbing}      % suppress blank line after tabbing

    \begin{itemize}
        \item Developed an image matching and recognition feature for a
        photo-printer Android app.  An 85\% recognition accuracy  was achieved. The application uses algorithms from
        computer vision as well as machine learning to extract features from the  photograph using mobile
        camera and 	search a match from an existing database. The program uses techniques like SIFT-like feature matching, SVM, and other various computer vision methodologies. 	
        \item Developed
        a python based basic image analysis application. The application
        segments natural  images into different components. It  also does
        face detection, pedestrian 	detection and segmentation of  images
        into super-pixels.

        \item Designed and developed an app feature that scans and recognizes
        simple logos using OCR and machine learning techniques. The app also performs pre-processing processing tasks like segmentation, circle detection etc. 

\end{itemize}

   \begin{tabbing}
   \hspace{2.3in}\= \hspace{2.6in}\= \kill % set up two tab positions
    {\bf Assistant Professor} \> \>July 2012 - Aug 2014\\
                     RGIT.\>     \>Andheri, Mumbai.
   \end{tabbing}
   \begin{itemize}
\item Worked on  two image processing based projects: \textit{Gesture Recognition}, \textit{Head Tracking}. The Gesture Recognition 
uses ideas from PCA based eigenfaces used for face recognition.  The Head Tracking project used Viola-Jones face detector. The output of the face detector is then used to change the viewing angle of a simple augmented cube on the screen.  
  
\item Conducted courses and designed  lab experiments/assignments for courses
like DSP, Image Processing and Random Signal Analysis. These experiments included simulation of random processes,
 implementation of image processing algorithms like filtering, image transforms etc.  

\item Introduced the use of OpenCV for Image Processing laboratory. Prior to this Matlab was used for image processing laboratory. I promoted and introduced OpenCV for image processing experiments. 
\end{itemize}

   \begin{tabbing}%
   \hspace{2.3in}\= \hspace{2.6in}\= \kill % set up two tab positions          
   {\bf Research Scholar}  \>  \> July 2008 - May 2012\\
                IIT-B\>    \>Powai, Mumbai.
   \end{tabbing}
 \begin{itemize}

\item Multiple low resolution images of the same scene are fused to super-resolve or obtain
high resolution image of the scene. This technique is called Super-Resolution. The method outperforms simple interpolation techniques in terms of  artifact removal, noise and PSNR. When applied to SDTV video to HDTV conversion, results were found to be better than simple resizing or interpolation. 

\item  Used optical-flow for  the SR-technique to  free the process  from the traditional global motion assumption. This enabled to do a point based-registration  and discard a global translation/motion assumption.
\end{itemize}

\section{EDUCATION}          
\begin{description}
 \item    Research Scholar, IIT-B, Mumbai. 
  \item  M.Tech (ICT) from DA-IICT, Gandhinagar.  
   \item     B.E (E.C) Saurashtra University, Gujarat.   
\end{description}    

\section{TECHNICAL SKILLS}          
\begin{description}
\item [Programming      :] Python, C/C++, Linux Scripting 
\item [Key-Areas        :] Machine-Learning/AI, IOT, Computer Vision. 
\end{description}


\section{PUBLICATION}
\begin{description}
\item   Deven Patel, Subhasis Chaudhuri: Performance Analysis for Image Super-Resolution Using Blur as a Cue. IEEE-ICAPR 2009
%\item \textit{Submitted} : SDTV to HDTV conversion using Optical-flow to IEEE Transactions on Consumer Electronics.
\end{description}



%\section{CONTACT}
%\begin{description}
%\item 
%\begin{tabular}{ll}%DOB &: 26-03-1982\\
%Address&: At.Po Kotlaw ``Abilasha'' \\&\hspace{7pt}Killa Pardi, Dist Valsad;\\&\hspace{7pt}Gujarat.\\&\hspace{7pt}PIN-396125\\
% deven.patel2@gmail.com \\
% +919819482499
%\end{tabular}
%\end{description} 
 
\end{resume}
\end{document}
